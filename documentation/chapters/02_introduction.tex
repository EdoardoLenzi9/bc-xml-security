\chapter{Introduction - Model capture}


The program implements XML Security standards using Bouncy Castle.
Verification XML signature syntax using such algorithms:

x509 certificate
GOST algorithm

with a custom signature method
 
Encrypting and decrypting XML documents with algorithms:
                          	Symmetric algorithms:
                                         	http://www.w3.org/2001/04/xmlenc#des-cbc
                                         	http://www.w3.org/2001/04/xmlenc#tripledes-cbc
                                         	http://www.w3.org/2001/04/xmlenc#aes128-cbc
http://www.w3.org/2001/04/xmlenc#aes192-cbc
http://www.w3.org/2001/04/xmlenc#aes256-cbc
                          	Symmetric algorithms with key wrap:
                                         	http://www.w3.org/2001/04/xmlenc#kw-tripledes
http://www.w3.org/2001/04/xmlenc#kw-aes128
http://www.w3.org/2001/04/xmlenc#kw-aes192
http://www.w3.org/2001/04/xmlenc#kw-aes256
                          	Algorithms using RSA with and without OAEP









The aim of this chapter is to 

What are the main features of the project?
What is bouncy castle?
























\section{}

reverse engineering of the source code by applying the initial understanding 
and detailed model capture patterns. 

apply the patterns 
* "Read all the Code in One Hour", 
* "Do a Mock Installation",  
* "Speculate about Design", 
* "Study the Exceptional Entities". 

Find out:
* main features?
* important source code entities?
* first impression of the quality of the design and implementation (also think of documentation, tests, etc.)?
* Do you think a reengineering is feasible?
* What are the exceptional packages, classes, and methods?
* inheritance structure look like?

apply the patterns 
* "Step through the Execution",
* "Look for the Contracts", 
* "Look for Key Methods", 
* "Look for Template/Hook Methods". 

How three main features of the subject system are implemented 
(mention the main classes, methods and briefly describe how they depend on each other).

Plot:
* Structural Analysis for Java (IBM)
* X-Ray
* Code City
* Your tool (use your favorite reverse engineering tool)

main features, ui, project goals (measurable goals)
Most valuable first (customer values/business model)
One hour overview
    * tests, abstract classes, comments, high fan-out classes
    * dependencies, version numbers
Skim doc (no doc)
Demo and mock installation
Test evolution
cost estimation

Project scope (description, context, goals, verification criteria)
Opportunities (what helps to achieve the project goals)
Risks (what can cause the main problems)

\begin{comment}
\chapter{Reverse Engineering}

Is the process of analyzing a subject system to identify the system components and their 
interrelationships and create representations of the system in another form or at higher 
level of abstraction.

It is an iterative process

* Persistent data
* Draw classes (sort by size/loc) color 
* Contracts
* Git history (pulsars)


---------------

Problem detection
Requirement analysis
Problem resolution

# Tests

Write tests
TDD 
\end{comment}

signedxml -> labels
utils
signedinfo
signedxml nested classes

dry ancestornamespace xmlattributenestedspace